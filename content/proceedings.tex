\section{Introduction}
The standard model of particle physics
(SM) is one of the most precise theories to this date.
Although the SM describes the interaction of the fundamental particles
with enormous precision, it is known to have its limitations.
For example it can not explain the matter-antimatter asymmetry in the universe,
nor does it include a description of interactions at the Planck-scale.
Extensions of the SM including new interactions and particles are called
\enquote{new physics} (NP).
In the SM the three leptons
$e, \mu, \tau$ only differ by their mass.
In all other properties they are interchangeable,
because they transform the same under the $\symup{SU(3)\times SU(2)\times U(1)}$
gauge group of the SM.
This Lepton Flavor Universality (LFU) can be viewed
as a key prediction of the SM.

Recent measurements of the LHCb experiment hint at deviations from LFU, which would be a clear sign of NP effects.
There are theoretical models to explain LFU in beyond standard model processes, such as lepto-quarks (LQ) which
directly couple leptons to quarks.

\subsection{$B^+\to K^+l^{+} l^{-}$ decays}
In the SM some decay rates can be predicted with high accuracy
The $B^+\to K^+l^{+} l^{-}$ and similar decays happen with $b\to s l^+l^-$
quark transitions. They contain a hadron in the final state and
are forbidden at tree level.
The lowest order diagram contributing in the SM is a
penguin diagram as seen in \autoref{fig:SMbkll}.
For that reason the branching fraction of that process
is suppressed on the order of
$\mathcal{O}(10^{-6})$ \cite{Zyla:2020zbs}.
This is good to observe NP effects,
since there are NP models, where this decay can happen
on tree-level.
An example of such a NP process is the
lepto-quark mediated diagram in \autoref{fig:BSMbkll}.

Similar processes with $b\to sl^+ l^-$ transitions,
allow similar analysis.
They provide different observables than
similar measurements without a meson in the final state.
\begin{figure}
	\centering
	\begin{tikzpicture}
		\begin{feynman}
			\vertex (b1) {$\overline b$};
			\vertex [right=of b1] (b2);
			\vertex [right=of b2] (b3);
			\vertex [right=of b3] (b4) {$\overline s$};

			\vertex [above=.8 of b1] (u1) {$u$};
			\vertex [above=.8 of b4] (u2) {$u$};

			\vertex at ($(b2)!0.5!(b3)!1!-90:(b3)$) (g1);
			\vertex [below=1.5 of b3] (g2);
			\vertex [below=1.2 of b4] (l1) {\(\ell^{+}\)};
			\vertex [below=.6 of l1] (l2) {\(\ell^{-}\)};

			\diagram* {
			(u1) -- [fermion] (u2),
			(b1) -- [anti fermion] (b2) -- [boson, edge label={$W^+$}] (b3) -- [anti fermion] (b4),
			(b2) -- [anti fermion, bend right] (g1) -- [anti fermion, bend right] (b3),
			(g1) -- [photon, bend right, edge label'={$\gamma/Z^0$}] (g2),
			(l1) -- [fermion, bend right] (g2) -- [fermion, bend right] (l2),
			};

		\end{feynman}
		\draw[decorate, decoration={brace, amplitude=5pt},line width=1pt](b1.south west) --node[left] {$B^+ \ $} (u1.north west);
		\draw[decorate, decoration={brace, amplitude=5pt},line width=1pt](u2.north east) --node[right] {$\ K^+ \ $} (b4.south east);
		\node [above=0.15 of g1]  {$\overline u, \overline c, \overline t$};
	\end{tikzpicture}
	\caption{\label{fig:SMbkll} SM lowest order diagram of the $B^+\to~K^+l^{+}~l^{-}$ decay.}%
\end{figure}

\begin{figure}
	\centering
	\begin{tikzpicture}
		\begin{feynman}
			\vertex (b1) {$\overline b$};
			\vertex [right=of b1] (b2);
			\vertex [right=of b2] (b3);
			\vertex [right=of b3] (b4) {$\overline s$};

			\vertex [above=.8 of b1] (u1) {$u$};
			\vertex [above=.8 of b4] (u2) {$u$};

			\vertex [below=1.2 of b4] (l1) {\(\ell^{+}\)};
			\vertex [below=.7 of l1] (l2) {\(\ell^{-}\)};

			\diagram* {
			(u1) -- [fermion] (u2),
			(b1) -- [anti fermion] (b2) -- [scalar, edge label={$LQ$}] (b3) -- [anti fermion] (b4),
			(l1) -- [fermion, bend left] (b3),
			(l2) -- [fermion, bend left] (b2),
			};

		\end{feynman}
		\draw[decorate, decoration={brace, amplitude=5pt},line width=1pt](b1.south west) --node[left] {$B^+ \ $} (u1.north west);
		\draw[decorate, decoration={brace, amplitude=5pt},line width=1pt](u2.north east) --node[right] {$\ K^+ \ $} (b4.south east);
	\end{tikzpicture}
	\caption{\label{fig:BSMbkll} NP example of the $B^+\to~K^+l^{+}~l^{-}$ decay with lepto-quark interaction.}%
\end{figure}





\section{Lepton Flavor Universality Tests}
In the last decade multiple tensions with SM predictions were observed in $b\to s l^+l^-$ transitions.
The deviation in observables from SM predictions, are called \enquote{flavor anomalies}.
These deviations recently reached the $\num{3}\sigma$ level, based on proton-proton collision data
at the LHCb detector.
Since the calculations in quantum chromo dynamics (QCD) rely on lattice-qcd predictions, because
of the non-pertubative nature of QCD,
the resulting uncertainties in QCD effects are high.
Since changing leptons does not affect QCD calculations because of LFU,
it is possible to define observables, which are independent from QCD uncertanties,
since they cancel. One class of theoretically clean observables is \cite{Hiller2004}
\begin{equation}
	R_{H}:= \frac{\int^{q_{\mathrm{max}}^2}_{q_{\mathrm{max}}^2}\frac{\dif \mathcal{B}(B\to H\mu^+\mu^-)}{\dif q^2} \dif q^2}{
		\int^{q_{\mathrm{max}}^2}_{q_{\mathrm{max}}^2}\frac{\dif \mathcal{B}(B\to He^+e^-)}{\dif q^2} \dif q^2
	} \overset{\mathrm{SM}}{\approx} 1.
\end{equation}
$\mathcal{B}$ is the branching ratio of the given process, where $H$ can be a hadron ($H=K$ in our case)
or even a collection of particles. The differential branching fractions are 
integrated over the invariant mass squared of the two leptons 
in the \emph{same} region $q^2~\in~(q_{\mathrm{min}}^2,~q_{\mathrm{max}}^2)$.
These observables are free from QCD uncertanties.
For example in the SM 
	$R_{K}=\num{1.00030}_{-\num{0.00007}}^{+\num{0.00010}}$
for $q^2\in(1, 6)\si{\giga\electronvolt}$
 has a relative uncertainty in the order of
$\mathcal{O}(10^{-4})$\cite{Hiller}. 
